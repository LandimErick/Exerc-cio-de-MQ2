\documentclass[
% -- opções da classe memoir --
12pt,				% tamanho da fonte
openright,			% capítulos começam em pág ímpar (insere página vazia caso preciso)
oneside,			% para impressão em recto e verso. Oposto a twoside
a4paper,			% tamanho do papel. 
% -- opções da classe abntex2 --
%chapter=TITLE,		% títulos de capítulos convertidos em letras maiúsculas
%section=TITLE,		% títulos de seções convertidos em letras maiúsculas
%subsection=TITLE,	% títulos de subseções convertidos em letras maiúsculas
%subsubsection=TITLE,% títulos de subsubseções convertidos em letras maiúsculas
% -- opções do pacote babel --
english,			% idioma adicional para hifenização
french,				% idioma adicional para hifenização
spanish,			% idioma adicional para hifenização
brazil				% o último idioma é o principal do documento
]{abntex2}

% ---
% Pacotes básicos 
% ---
\usepackage{ulem}
%\usepackage{natbib}
\usepackage{amssymb}
\usepackage{mathrsfs}
\usepackage{tikz}
\usepackage{graphicx}
\usepackage{cancel}
\usepackage{epsfig}
%\usepackage{float}
\usepackage[compat=1.1.0]{tikz-feynman}
\usepackage{physics}
\usepackage{lmodern}			% Usa a fonte Latin Modern			
\usepackage[T1]{fontenc}		% Selecao de codigos de fonte.
\usepackage[utf8]{inputenc}		% Codificacao do documento (conversão automática dos acentos)
\usepackage{indentfirst}		% Indenta o primeiro parágrafo de cada seção.
\usepackage{color}				% Controle das cores
\usepackage{graphicx}			% Inclusão de gráficos
\usepackage{microtype} 			% para melhorias de justificação

% compila o indice
% ---
\makeindex
% ---

% ----
% Início do documento
% ----
\begin{document}
\chapter{Exercício 2 parte 2}
 Seja o estado térmico 
\begin{equation}
	\rho = N e^{-\beta H_{AB}} \in \mathcal{H}_{AB} = \mathcal{H}_A\otimes\mathcal{H}_{B}
\end{equation}
e o hamiltoniano
\begin{equation}
	H_{AB}=J \vec{\sigma}_A\cdot\vec{\sigma}_B
\end{equation}
Calcule 
\begin{equation}
	\langle CHSH\rangle = tr(\rho CHSH)
\end{equation}
Onde CHSH é o operador de Bell, 
\begin{equation}
	CHSH = AB+A'B+AB'-A'B'
\end{equation}
em que $[A_i,B_j]=0$, $[A_i,A_j]=[B_i,B_j]\neq0$, $A^2=1$, $B^2=1$.
\section{Solução: }
\subsection{Autoestados do Hamiltoniano}
Partindo da equação de Schrödinger, 
\begin{equation}
	H_{AB}\ket{\psi}_{AB}=E_n\ket{\psi}_{AB} 
\end{equation}
e considerando um estado genérico no espaço de Hilbert $\mathcal{H}_{AB}$
\begin{equation}
	\ket{\psi}_{AB}= \sum_i\sum_j C_{ij} \ket{i}\ket{j};\qquad i,j \in \{0,1\}
\end{equation}
Vamos, a partir da equação de Schrödinger, encontrar os autovalores e autovetores. 
\begin{multline}
	J(\sigma_x^A\sigma_x^B +\sigma_y^A\sigma_y^B + \sigma_z^A\sigma_z^B)(C_{00} \ket{0}_A\ket{0}_B+ C_{10} \ket{1}_A\ket{0}_B +C_{01}\ket{0}_A\ket{1}_B+C_{11} \ket{1}_A\ket{1}_B)\\= E_n(C_{00} \ket{0}_A\ket{0}_B+ C_{10} \ket{1}_A\ket{0}_B +C_{01}\ket{0}_A\ket{1}_B+C_{11} \ket{1}_A\ket{1}_B)
\end{multline}
Temos portanto,
\begin{multline}
	J\sigma_x^A\sigma_x^B (C_{00} \ket{0}_A\ket{0}_B+ C_{10} \ket{1}_A\ket{0}_B +C_{01}\ket{0}_A\ket{1}_B+C_{11} \ket{1}_A\ket{1}_B) + \\
	J\sigma_y^A\sigma_y^B (C_{00} \ket{0}_A\ket{0}_B+ C_{10} \ket{1}_A\ket{0}_B +C_{01}\ket{0}_A\ket{1}_B+C_{11} \ket{1}_A\ket{1}_B) + \\
    J\sigma_z^A\sigma_z^B (C_{00} \ket{0}_A\ket{0}_B+ C_{10} \ket{1}_A\ket{0}_B +C_{01}\ket{0}_A\ket{1}_B+C_{11} \ket{1}_A\ket{1}_B) = \\
    E_n(C_{00} \ket{0}_A\ket{0}_B+ C_{10} \ket{1}_A\ket{0}_B +C_{01}\ket{0}_A\ket{1}_B+C_{11} \ket{1}_A\ket{1}_B)
\end{multline}

Ou seja, considerando que os estados da base $\{\ket{0},\ket{1}\}$ são autoestados de $\sigma_z$,
\begin{multline}
	J(C_{00} \ket{1}_A\ket{1}_B+ C_{10} \ket{0}_A\ket{1}_B +C_{01}\ket{1}_A\ket{0}_B+C_{11} \ket{0}_A\ket{0}_B) + \\
	J(C_{00} i^2 \ket{1}_A\ket{1}_B -C_{10} i^2 \ket{0}_A\ket{1}_B -C_{01}i^2\ket{1}_A\ket{0}_B+C_{11}(-i)^2\ket{0}_A\ket{0}_B) + \\
	J(C_{00} \ket{0}_A\ket{0}_B- C_{10} \ket{1}_A\ket{0}_B -C_{01}\ket{0}_A\ket{1}_B+C_{11} \ket{1}_A\ket{1}_B) = \\
	E_n(C_{00} \ket{0}_A\ket{0}_B+ C_{10} \ket{1}_A\ket{0}_B +C_{01}\ket{0}_A\ket{1}_B+C_{11} \ket{1}_A\ket{1}_B)
\end{multline}
 	Agrupando os termos 
 	 	\begin{multline}
 		JC_{11}\ket{1}_A\ket{1}_B + (2J C_{10} -JC_{01})\ket{0}_A\ket{1}_B + (2J C_{01} -JC_{10})\ket{1}_A\ket{0}_B +  JC_{0}\ket{0}_A\ket{0}_B \\= E_n(C_{00} \ket{0}_A\ket{0}_B+ C_{10} \ket{1}_A\ket{0}_B +C_{01}\ket{0}_A\ket{1}_B+C_{11} \ket{1}_A\ket{1}_B)
 	\end{multline}
 	Ou seja, temos o seguinte sistema de equações 
\begin{equation}
    \begin{cases}
		JC_{11} = E_nC_{11}\\
		C_{10} =\frac{(J +E_n)}{2J}C_{01}\\
		C_{01} =\frac{(J +E_n)}{2J}C_{10}\\
		JC_{00}= E_NC_{00}\\
	\end{cases}
\end{equation}

Ou seja, 
\begin{equation}
C_{10} =\frac{(J +E_n)^2}{4J^2}C_{10}	
\end{equation}
Ou seja,
\begin{equation}
C_{10} = \pm  C_{01}~e~(J +E_n)^2=4J^2  \rightarrow -3J^2 +2JE_n +E_n^2 =0
\end{equation}
Resolvendo a equação de segundo grau para $E_n$
\begin{equation}
	E_n= -J \pm 2J  \Rightarrow  E_1= -3J\quad e \quad E_2=J.
\end{equation}

Se $E_n=-3J$, temos 
\begin{equation}
	\begin{cases}
	JC_{11} = -3JC_{11}\\
	JC_{10} = -JC_{01}\\
    JC_{01} =-JC_{10}\\
	JC_{00}= -3JC_{00}\\
  \end{cases}
\end{equation}
Ou seja, $C_{11}=C_{00}=0$ e $C_{01} =-C_{10}=\frac{1}{\sqrt{2}}$, e portanto temos o seguinte autoestado:
\begin{equation}
	\ket{\psi_n}_{AB}= \frac{1}{\sqrt{2}}(\ket{1}\ket{0}- \ket{0}\ket{1})   
\end{equation}  

Se $E_n=J$, temos 
\begin{equation}
	\begin{cases}
	C_{11} = C_{11}\\
	C_{10} =C_{01}\\
	C_{01} =C_{10}\\
	C_{00}= C_{00}\\
  \end{cases}
\end{equation}
Ou seja, 
\begin{equation}
	\ket{\psi_n}_{AB}= C_{00}\ket{0}\ket{0} + C_{11}\ket{1}\ket{1}) + C_{10}(\ket{1}\ket{0}+\ket{0}\ket{1})   
\end{equation}
Quaisquer estados que respeitem esta forma são autoestados do hamiltoniano. Em particular podemos escrever $C_{10}=0$ e $C_{00}=C_{11}= 1/\sqrt{2}$, e portanto,
\begin{equation}
	\ket{\psi_2}_{AB}= \frac{1}{\sqrt{2}}(\ket{1}\ket{1}+ \ket{0}\ket{0})   
\end{equation}  
Escrevendo $C_{00}=-C_{11}= 1/\sqrt{2}$
\begin{equation}
	\ket{\psi_3}_{AB}= \frac{1}{\sqrt{2}}(\ket{1}\ket{1}- \ket{0}\ket{0})   
\end{equation}  
Escrevendo $C_{00}=C_{11}=0$, $C_{10}= 1/\sqrt{2}$
\begin{equation}
	\ket{\psi_4}_{AB}= \frac{1}{\sqrt{2}}(\ket{1}\ket{0}+ \ket{0}\ket{0})   
\end{equation}  
Como estes são os autovalores do hamiltoniano, temos que esta é uma base para o espaço de Hilbert $\mathcal{H}_{AB}$. Em particular, estes estados são precisamente os estados de Bell, compostos pelo singleto $\ket{\psi_1}$ e o tripleto $\ket{\psi_2},~\ket{\psi_3},~\ket{\psi_4}$. 
 \subsection{Matriz densidade}
 
 Voltando para a matriz densidade do estado térmico, $\rho_\beta=e^{-\beta H_{AB}}$ e lembrando que $\mathbb{I}=\sum_n\ket{\psi_n}\bra{\psi_n}$ 
 \begin{equation}
 	\rho_\beta=\sum_ne^{-\beta H_{AB}}\ket{\psi_n}\bra{\psi_n}
 \end{equation}
 Usando que $H_{AB}\ket{\psi}_{AB}=E_n\ket{\psi}_{AB}$
 \begin{equation}
	\rho_\beta=\sum_ne^{-\beta E_n}\ket{\psi_n}\bra{\psi_n}.
\end{equation}
Como $tr \rho=1$, temos que 
 \begin{equation}
	\rho_\beta=\frac{\sum_ne^{-\beta E_n}\ket{\psi_n}\bra{\psi_n}}{\sum_je^{-\beta E_j}}.
\end{equation}
Ou seja,
\begin{equation}
	\rho_\beta=\frac{e^{3\beta J}\ket{\psi_1}\bra{\psi_1}+e^{-\beta J}\ket{\psi_2}\bra{\psi_2}+e^{-\beta J}\ket{\psi_3}\bra{\psi_3}+e^{-\beta J}\ket{\psi_4}\bra{\psi_4}}{e^{3\beta J}+3e^{-\beta J}}.
\end{equation} 
\subsection{Cálculo da desigualdade de Bell}
Queremos Calcular o valor esperado do operador de Bell $C=AB+A'B+AB'-A'B'$. Vamos utilizar que 
\begin{equation}
	\langle C \rangle =tr(\rho C)=\sum_n \bra{\psi_n}\rho C\ket{\psi_n}
\end{equation}
 \begin{equation}
 	\langle C \rangle =\sum_n \bra{\psi_n}\left(\frac{e^{3\beta J}\ket{\psi_1}\bra{\psi_1}+e^{-\beta J}\ket{\psi_2}\bra{\psi_2}+e^{-\beta J}\ket{\psi_3}\bra{\psi_3}+e^{-\beta J}\ket{\psi_4}\bra{\psi_4}}{e^{3\beta J}+3e^{-\beta J}}\right) \ket{\psi_n}
 \end{equation}
 Ou seja,
 \begin{multline}
 	\bra{\psi_1}\left(\frac{e^{3\beta J}\ket{\psi_1}\bra{\psi_1}+e^{-\beta J}\ket{\psi_2}\bra{\psi_2}+e^{-\beta J}\ket{\psi_3}\bra{\psi_3}+e^{-\beta J}\ket{\psi_4}\bra{\psi_4}}{e^{3\beta J}+3e^{-\beta J}}\right) C \ket{\psi_1} +\\
 	\bra{\psi_2}\left(\frac{e^{3\beta J}\ket{\psi_1}\bra{\psi_1}+e^{-\beta J}\ket{\psi_2}\bra{\psi_2}+e^{-\beta J}\ket{\psi_3}\bra{\psi_3}+e^{-\beta J}\ket{\psi_4}\bra{\psi_4}}{e^{3\beta J}+3e^{-\beta J}}\right) C \ket{\psi_2}+\\
 	\bra{\psi_3}\left(\frac{e^{3\beta J}\ket{\psi_1}\bra{\psi_1}+e^{-\beta J}\ket{\psi_2}\bra{\psi_2}+e^{-\beta J}\ket{\psi_3}\bra{\psi_3}+e^{-\beta J}\ket{\psi_4}\bra{\psi_4}}{e^{3\beta J}+3e^{-\beta J}}\right) C \ket{\psi_3}+ \\
 	\bra{\psi_4}\left(\frac{e^{3\beta J}\ket{\psi_1}\bra{\psi_1}+e^{-\beta J}\ket{\psi_2}\bra{\psi_2}+e^{-\beta J}\ket{\psi_3}\bra{\psi_3}+e^{-\beta J}\ket{\psi_4}\bra{\psi_4}}{e^{3\beta J}+3e^{-\beta J}}\right) C \ket{\psi_4}
 	\end{multline}
Como os estados Bell são ortonormais, temos 
\begin{equation}
\langle C \rangle=\frac{e^{3\beta J}\bra{\psi_1}C\ket{\psi_1}+e^{-\beta J}\bra{\psi_2}C\ket{\psi_2}+e^{-\beta J}\bra{\psi_3}C\ket{\psi_3}+e^{-\beta J}\bra{\psi_4}C\ket{\psi_4}}{e^{3\beta J}+3e^{-\beta J}}
\end{equation}
Considerando os seguintes operadores de Bell
\begin{equation}
	A\ket{0}=e^{i\alpha}\ket{1}; \qquad A\ket{1}=e^{-i\alpha}\ket{0} 
\end{equation}
\begin{equation}
	B\ket{0}=e^{i\gamma}\ket{1}; \qquad B\ket{1}=e^{-i\gamma}\ket{0} 
\end{equation}
Calculando $\bra{\psi_1}AB\ket{\psi_1}$
\begin{equation}
	\bra{\psi_1}AB\ket{\psi_1}= \frac{1}{2}(\bra{1}\bra{0}- \bra{0}\bra{1}) AB(\ket{1}\ket{0}- \ket{0}\ket{1})
\end{equation}
\begin{equation}
=\frac{1}{2}(\bra{1}\bra{0}AB\ket{1}\ket{0} - \bra{1}\bra{0}AB\ket{0}\ket{1} -\bra{0}\bra{1}AB\ket{1}\ket{0} + \bra{0}\bra{1}AB\ket{0}\ket{1})
\end{equation}
\begin{equation}
	=-\frac{1}{2}(e^{i(\alpha-\gamma)} +e^{-i(\alpha-\gamma)} )=-cos(\alpha-\gamma)
\end{equation}
Ou seja,
\begin{equation}
	\bra{\psi_1}C\ket{\psi_1}= -cos(\alpha-\gamma) -cos(\alpha'-\gamma) -cos(\alpha-\gamma') + cos(\alpha'-\gamma')
\end{equation}

Calculando $\bra{\psi_2}AB\ket{\psi_2}$
\begin{equation}
	\bra{\psi_2}AB\ket{\psi_2}= \frac{1}{2}(\bra{0}\bra{0}+ \bra{1}\bra{1}) AB(\ket{0}\ket{0}+ \ket{1}\ket{1})
\end{equation}
\begin{equation}
	=\frac{1}{2}(\bra{0}\bra{0}AB\ket{0}\ket{0} + \bra{0}\bra{0}AB\ket{1}\ket{1} +\bra{1}\bra{1}AB\ket{0}\ket{0} + \bra{1}\bra{1}AB\ket{1}\ket{1})
\end{equation}
\begin{equation}
	=\frac{1}{2}(e^{-i(\alpha+\gamma)} +e^{i(\alpha+\gamma)} )=cos(\alpha+\gamma)
\end{equation}
Ou seja,
\begin{equation}
	\bra{\psi_2}C\ket{\psi_2}= cos(\alpha+\gamma) +cos(\alpha'+\gamma) +cos(\alpha+\gamma') - cos(\alpha'+\gamma')
\end{equation}

Calculando $\bra{\psi_3}AB\ket{\psi_3}$
\begin{equation}
	\bra{\psi_3}AB\ket{\psi_3}= \frac{1}{2}(\bra{0}\bra{0}- \bra{1}\bra{1}) AB(\ket{0}\ket{0}- \ket{1}\ket{1})
\end{equation}
\begin{equation}
	=\frac{1}{2}(\bra{0}\bra{0}AB\ket{0}\ket{0} - \bra{0}\bra{0}AB\ket{1}\ket{1} -\bra{1}\bra{1}AB\ket{0}\ket{0} + \bra{1}\bra{1}AB\ket{1}\ket{1})
\end{equation}
\begin{equation}
	=-\frac{1}{2}(e^{-i(\alpha+\gamma)} +e^{i(\alpha+\gamma)} )=-cos(\alpha+\gamma)
\end{equation}
Ou seja,
\begin{equation}
	\bra{\psi_2}C\ket{\psi_2}= -cos(\alpha+\gamma) -cos(\alpha'+\gamma) -cos(\alpha+\gamma') + cos(\alpha'+\gamma')
\end{equation}




Calculando $\bra{\psi_4}AB\ket{\psi_4}$
\begin{equation}
	\bra{\psi_4}AB\ket{\psi_4}= \frac{1}{2}(\bra{1}\bra{0}+ \bra{0}\bra{1}) AB(\ket{1}\ket{0}+ \ket{0}\ket{1})
\end{equation}
\begin{equation}
	=\frac{1}{2}(\bra{1}\bra{0}AB\ket{1}\ket{0} + \bra{1}\bra{0}AB\ket{0}\ket{1} +\bra{0}\bra{1}AB\ket{1}\ket{0} + \bra{0}\bra{1}AB\ket{0}\ket{1})
\end{equation}
\begin{equation}
	=\frac{1}{2}(e^{i(\alpha-\gamma)} +e^{-i(\alpha-\gamma)} )=cos(\alpha-\gamma)
\end{equation}
Ou seja,
\begin{equation}
	\bra{\psi_4}C\ket{\psi_4}= cos(\alpha-\gamma) 
	+cos(\alpha'-\gamma) +cos(\alpha-\gamma') - cos(\alpha'-\gamma')
\end{equation}

Portanto,
\begin{multline}
	\langle C \rangle = \frac{-e^{3\beta J}(cos(\alpha-\gamma) +cos(\alpha'-\gamma) +cos(\alpha-\gamma') - cos(\alpha'-\gamma'))}{e^{3\beta J} +3e^{-\beta J}} \\
	+  \frac{e^{-\beta J}(cos(\alpha+\gamma) +cos(\alpha'+\gamma) +cos(\alpha+\gamma') - cos(\alpha'+\gamma'))}{e^{3\beta J} +3e^{-\beta J}} \\ + \frac{-e^{-\beta J}(cos(\alpha+\gamma) +cos(\alpha'+\gamma) +cos(\alpha+\gamma') - cos(\alpha'+\gamma'))}{e^{3\beta J} +3e^{-\beta J}} \\ + \frac{e^{-\beta J}(cos(\alpha-\gamma) +cos(\alpha'-\gamma) +cos(\alpha-\gamma') - cos(\alpha'-\gamma'))}{e^{3\beta J} +3e^{-\beta J}}
\end{multline}
Ou seja,

\begin{equation}
	\langle C \rangle = (cos(\alpha-\gamma) +cos(\alpha'-\gamma) +cos(\alpha-\gamma') - cos(\alpha'-\gamma'))\frac{(e^{-\beta J}-e^{3\beta J})}{e^{3\beta J} +3e^{-\beta J}}
\end{equation}
Considerando $\alpha=0;\alpha=\frac{\pi}{2};\gamma=-\frac{\pi}{4};\gamma=\frac{\pi}{4};$
\begin{equation}
	\langle C \rangle = 2\sqrt{2}\frac{(e^{-\beta J}-e^{3\beta J})}{e^{3\beta J} +3e^{-\beta J}}
\end{equation}
Podemos também escrever 
\begin{equation}
	\langle C \rangle = -2\sqrt{2}\frac{2sinh(2\beta J)}{e^{2\beta J} +3e^{-2\beta J}}
\end{equation}
\subsubsection{Violações da desigualdade}
Quando $\beta\rightarrow\infty$, $\langle C \rangle \rightarrow 2\sqrt{2}$, ou seja, a desigualdade de Bell tende a violação máxima à medida que a temperatura do estado térmico tende a zero. Em contrapartida, ou seja, quando  a tende ao infinito a desigualdade tende a zero, ou seja, para $\beta\rightarrow0$, $\langle C \rangle \rightarrow0$. Portanto, o aumento de temperatura destrói o emaranhamento do estado. 

 Vamos investigar a partir de qual valor da temperatura ocorre a violação, ou seja, $|\langle C\rangle| >2$
\begin{equation}
	 2\sqrt{2}\frac{(-e^{-\beta J}+e^{3\beta J})}{e^{3\beta J} +3e^{-\beta J}}> 2
\end{equation} 
Ou seja, 
\begin{equation}
	\sqrt{2}(-e^{-\beta J}+e^{3\beta J}) >(e^{3\beta J} +3e^{-\beta J})  
\end{equation}
Reescrevendo temos,
\begin{equation}
	e^{4\beta J}  >4\sqrt{2} +2 
\end{equation}
ou ainda 
\begin{equation}
	\beta J  >\frac{\ln(4\sqrt{2} +2)}{4} 
\end{equation}

\end{document}

